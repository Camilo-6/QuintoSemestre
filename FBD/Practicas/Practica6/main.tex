\documentclass{article}
\usepackage{graphicx} % Required for inserting images

\title{Practica5}
\author{José Camilo García Ponce}

\begin{document}

Practica 6 \\
García Ponce José Camilo \\

\begin{itemize}
    \item Ejercicio 1 \\
    ¿Cuál es la llave para R? Normalizar a R en la forma 2NF y luego en la forma 3NF. \\
    \begin{itemize}
        \item Considera la relación $R = \{A, B, C, D, E, F, G, H, I, J\}$ y el conjunto de dependencias funcionales $F = \{ \{A, B\} \rightarrow \{C\}, \{A\} \rightarrow \{D, E\}, \{B\} \rightarrow \{F\}, \{F\} \rightarrow \{G,H\}, \{D\} \rightarrow \{I, J\} \}$. \\

        La llave es $\{A,B\}$, ya que $A$ determina a $D$, $E$ y $A$, $B$ determina a $F$ y $B$, $A$ y $B$ determinan a $C$, $F$ determina a $G$ y $H$, $D$ determina a $I$ y $J$ \\

        La forma 2NF es $R_1 = \{A,B,C\}$, $R_2 = \{A,D,E,I,J\}$ y $R_3 = \{B,F,G,H\}$, con $\{A,B\}$ la llave de $R_1$, $\{A\}$ la llave de $R_2$ y $\{B\}$ la llave de $R_3$, creo que si cumple 2NF pero la llave de $R_2$ y $R_3$ no determinan a todos los atributos, por eso es mejor la forma de abajo \\

        La forma 3NF es $R_1 = \{A,B,C\}$, $R_2 = \{A,D,E\}$, $R_3 = \{B,F\}$, $R_4 = \{F,G,H\}$ y $R_5 = \{D,I,J\}$ con $\{A,B\}$ la llave de $R_1$, $\{A\}$ la llave de $R_2$, $\{B\}$ la llave de $R_3$, $\{F\}$ la llave de $R_4$ y $\{D\}$ la llave de $R_5$ \\

        \item Considera la misma relación $R$ con conjunto de dependencias funcionales $G = \{\{A, B\} \rightarrow \{C\}, \{B, D\} \rightarrow \{E, F\}, \{A, D\} \rightarrow \{G, H\}, \{A\} \rightarrow \{I\}, \{H\} \rightarrow \{J\} \}$. \\

        La llave es $\{A,B,D\}$, ya que $A$ determina a $I$ y $A$, $B$ determina a $B$, $A$ y $B$ determinan a $C$, $D$ determina a $D$, $B$ y $D$ determinan a $E$ y $F$, $A$ y $D$ determinan a $G$ y $H$, $H$ determina a $J$ \\

        La forma 2NF es $R_1 = \{A,B,D\}$, $R_2 = \{A,B,C\}$, $R_3 = \{B,D,E,F\}$, $R_4 = \{A,D,G,H,J\}$ y $R_5 = \{A,I\}$, con $\{A,B,D\}$ la llave de $R_1$, $\{A,B\}$ la llave de $R_2$, $\{B,D\}$ la llave de $R_3$, $\{A,D\}$ la llave de $R_4$ y $\{A\}$ la llave de $R_5$, creo que si cumple 2NF pero la llave de $R_2$ y $R_3$ no determinan a todos los atributos, por eso es mejor la forma de abajo \\

        La forma 3NF es $R_1 = \{A,B,D\}$, $R_2 = \{A,B,C\}$, $R_3 = \{B,D,E,F\}$, $R_4 = \{A,D,G,H\}$, $R_5 = \{A,I\}$ y $R_6 = \{H,J\}$, con $\{A,B,D\}$ la llave de $R_1$, $\{A,B\}$ la llave de $R_2$, $\{B,D\}$ la llave de $R_3$, $\{A,D\}$ la llave de $R_4$, $\{A\}$ la llave de $R_5$ y $\{H\}$ la llave de $R_6$ \\
        
    \end{itemize}

    \item Ejercicio 2 \\
    \begin{itemize}
        \item ¿Cuál de las siguientes dependencias podrían mantenerse en la relación de arriba? Si la dependencia no se mantiene, explica porqué especificando las tuplas que causan la violación: \\
        \begin{enumerate}
            \item $A \rightarrow B$ \\
            No se mantiene, ya que la tupla 1 tiene 10 en $A$ y b1 en $B$, pero la tupla 2 tiene 10 en $A$ y b2 en $B$ \\

            \item $B \rightarrow C$ \\
            Si se mantiene \\

            \item $C \rightarrow B$ \\
            No se mantiene, ya que la tupla 1 tiene c1 en $C$ y b1 en $B$, pero la tupla 3 tiene c1 en $C$ y b4 en $B$ \\

            \item $B \rightarrow A$ \\
            No se mantiene, ya que la tupla 1 tiene b1 en $B$ y 10 en $A$, pero la tupla 5 tiene b1 en $B$ y 13 en $A$ \\

            \item $C \rightarrow A$ \\
            No se mantiene, ya que la tupla 1 tiene c1 en $C$ y 10 en $A$, pero la tupla 5 tiene c1 en $C$ y 13 en $A$ \\
        \end{enumerate}

        \item ¿La relación tiene algún candidato para la llave primaria? En caso de tener, explicar cuál y porqué; si no tiene explicar porqué. \\
        La llave primaria trivial seria usar $\{Tuple\#\}$, pero si la ignoramos la llave primaria puede ser $\{A,B\}$, debido a que como vimos arriba $B \rightarrow C$, entonces podemos determinar a $C$ con base en $B$ y ya solo nos faltaría $A$ para hacer que fuera única, además ninguna dos tuplas tienen valores iguales en $A$ y $B$ \\

    \end{itemize}

    \item Ejercicio 3 \\
    Considera la siguiente relación de libros publicados: \\
    $BOOK (Book\_title , Author\_name , Book\_type , List\_price , Author\_affil, Publisher )$ \\
    $Author\_affil$ se refiere a la afiliación del autor, es decir la institución a la que pertenece. \\
    Supón que las siguientes dependencias existen: \\
    $Book\_title \rightarrow Publisher , Book\_type$ \\
    $Book\_type \rightarrow List\_price$ \\
    $Author\_name \rightarrow Author\_affil$ \\
    \begin{itemize}
        \item ¿En qué forma normal está la relación inicialmente? Explica tu respuesta. \\
        Esta en forma 1NF, ya que la llave primaria es $\{Book\_title, Author\_name\}$, ya que no tenemos atributos multivaluados o compuestos, pero tenemos dependencias funcionales parciales, $Book\_title \rightarrow Publisher , Book\_type$ y $Author\_name \rightarrow Author\_affil$, entonces no cumple 2NF y por lo tanto tampco 3NF \\

        \item Aplica la normalización hasta que ya no puedas descomponer más las relaciones resultantes. Establece las razones detrás de cada descomposición. \\
        Primero descomponemos en \\
        $BOOK (Book\_title, Author\_name)$ \\
        $BOOK_2(Book\_title, Book\_type, List\_price, Publisher)$ \\
        $BOOK_3(Author\_name, Author\_affil)$ \\
        de esta forma ya no tenemos dependencias funcionales parciales y ahora si cumplimos la 2NF, pero tenemos una dependencia funcional transitiva con $Book\_title \rightarrow Book\_type$ y $Book\_type \rightarrow List\_price$, por lo tanto no cumplimos forma 3NF \\
        
        Entonces descomponemos en \\
        $BOOK (Book\_title, Author\_name)$ \\
        $BOOK_2(Book\_title, Book\_type, Publisher)$ \\
        $BOOK_3(Author\_name, Author\_affil)$ \\
        $BOOK_4(Book\_type, List\_price)$ \\
        de esta forma ya no tenemos dependencias funcionales transitivas y ahora si cumplimos la 3NF \\
    \end{itemize}
    
\end{itemize}

\end{document}
